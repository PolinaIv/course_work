\chapter{Основные понятия}\label{chp:main_definitions}
\textbf{ \textit{Временная Иерархическая Память}}, или \textbf{\textit{Hierarchical Temporal Memory (HTM)}} --- это технология машинного обучения, направленная на  создание(обобщение) структурных и алгоритмических свойств неокортекса. [2] Разработка HTM осуществлялась компанией Numenta и продолжается и на сегодняшний день. Под основными алгоритмическими свойствами здесь понимаются распознавание (классификация) объектов, предсказание поведения объектов во времени и запоминание и др.
 
Далее будут приведены биологические термины. Пояснения по их использованию в сетях HTM приведено в Главе 1. 

\textbf{\textit{Неокортекс}} (<<новая кора>>) –- основная (по размеру) часть коры головного мозга.

\textbf{\textit{Нейрон}} --- нервные клетки. Основными частями нейрона являются клеточное тело, аксон и дендриты. Для сетей HTM за основу берется структура нейрона с одним аксоном и несколькими дендритами. 

\textbf{\textit{Аксон}} --- главный отросток нервной клетки, по которому она передает информацию следующей клетке в нейронной цепи. Если нейрон образует выходные связи с большим числом других клеток, его аксон многократно ветвится, чтобы сигналы могли дойти до каждой из них. [3]

\textbf{\textit{Синапс}} --- действительные места соединения между нейронами (специфические точки на поверхности нервных клеток, где происходит их контакт). [3]

\textbf{\textit{Дендриты}} ---  как правило, короткие и сильно разветвлённые отростки нейрона, служащие главным местом образования влияющих на нейрон возбуждающих и тормозных синапсов , и которые передают возбуждение к телу нейрона [wiki]. На дендритах и на поверхности центральной части нейрона, окружающей ядро , находятся входные синапсы, образуемые аксонами других нейронов.[3]